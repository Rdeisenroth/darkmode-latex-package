% \iffalse meta-comment
%
% Copyright (C) 2022 by Ruben Deisenroth <rdeisenroth@gmail.com>
%
% This file may be distributed and/or modified under the
% conditions of the LaTeX Project Public License, either
% version 1.3 of this license or (at your option) any later
% version. The latest version of this license is in:
%
% http://www.latex-project.org/lppl.txt
%
% and version 1.3 or later is part of all distributions of
% LaTeX version 2005/12/01 or later.
%
% \fi

% \iffalse
%<package>\NeedsTeXFormat{LaTeX2e}[2005/12/01]
%<package>\RequirePackage{expl3,l3keys2e}
%<package>\ProvidesExplPackage{darkmode}
%<package>{2022/02/08}
%<package>{1.0 snapshot}
%<package>{Provide general dark mode support for any LaTeX document.}
%<package>\RequirePackage{pagecolor}
%
%<*driver>
\documentclass{l3doc}
\hypersetup{
    hidelinks,
}
\usepackage{etoolbox}
% \usepackage[enable]{darkmode}
\EnableCrossrefs
\CodelineIndex
\RecordChanges
\setlength{\parindent}{0pt}
%\setlength{\parskip}{1em}%
\begin{document}
\DocInput{darkmode.dtx}
\end{document}
%</driver>
% \fi
% \CheckSum{0}
% \GetFileInfo{darkmode.sty}
% \title{The \pkg{darkmode} package\thanks{This document
% corresponds to \textsf{darkmode}~\fileversion,
% dated~\filedate.}}
% \author{Ruben Deisenroth \\ \texttt{rdeisenroth@gmail.com}}
% \maketitle
% \tableofcontents
%
% \begin{documentation}
% \section{Introduction}
% The \pkg{darkmode} Package provides an API for template and package developers to create dynamic color schemes for light- and darkmodes. For those unaware: We refer to \emph{dark mode} when a document has a dark background with a light font and to light mode if it has a dark font with a light background.
% \subsection{Why}
% Now you might think: Why would anyone even need to use a dark mode for PDF files?\\ While for the common task of printing a dark mode has a lot of disadvantages and would take a lot of ink, theese days it becomes more and more common to view PDF files directly on a Screen. And Screens emit a lot of Blue light, which - especially in the evening - can be really annoying to stare at when working on a large text document. Also modern displays can get so bright that a stark black and white contrast is too strong for most lighting scenarios.
% \section{Usage}

% \subsection{Basic usage}
% The basic Idea behind this package is that the dark mode is enabled in the preamble once, either by passing the package option \verb+enable+ or via the command \tn{enabledarkmode}, and applies to the rest of the document. 
% \subsection{Adopting the Darkmode}
% When defining custom colors and commands, we can check if the dark mode is active or not to dynamically adjust the Colors.\par
% Example: \begin{verbatim}
% \DeclareDocumentCommand{\bluecode}{m}{%
%     \texttt{\textcolor{\IfDarkModeTF{cyan}{blue}}{#1}}%
% }
% \end{verbatim}
% Note that in most scenarios it is possible to use the \tn{IfDarkModeTF} command inline without any sideeffects. If cases where this doesn't work, an alternative is to wrap the entire definition of a macro in the switch:
% \begin{verbatim}
% \IfDarkModeTF{%
%     \definecolor{accentcolor}{RGB}{0, 157, 129}%
% }{%
%     \definecolor{accentcolor}{RGB}{0, 122, 93}%
% }
% \end{verbatim} 
% Another usefull trick is the usage of the colors \cs{thepagecolor} from the \pkg{pagecolor}-Package and the color \texttt{fgcolor} which is defined to the default font color. You can mix any color with these two colors to easily derive fitting colors:
\begin{verbatim}
% % Fancy Altering Table Row Colors
% \rowcolors{2}{\thepagecolor}{fgcolor!10!\thepagecolor}
\end{verbatim}
Note that \verb+yellow!10!\thepagecolor+ is not the same as \verb+yellow!10+ because the second one would just default to \verb+yellow!10!white+.
% \section{Detailed documentation}
% \subsection{Package options}
% There following options can be passed to the package:
% \begin{itemize}
%    \item \texttt{enable} - boolean input, enables or disables the dark mode for the entire document
% \item \texttt{defaulthook} - boolean input, whether to enable or disable the default hook (\tn{@darkmode@on@enable} or \tn{\@darkmode@on@disable})
% \item \texttt{nodefaulthook} - meta key for disabeling the default hook
%\end{itemize}
% \subsection{User Commands}
% \begin{function}{\enabledarkmode,\disabledarkmode}
%    \tn{enabledarkmode} enables the dark Mode at the current position, while \tn{disabledarkmode} disables the dark mode at the current position.\\\textbf{Note:} that this is only intended to be used in the preamble for an entire document mainly with a CI/CD. Trying to toggle the dark mode between different pages may lead to unexpected behavior and is not recommended. For regular use to pass the \emph{enabled} option to the package.
% \end{function}
% \begin{function}[EXP]{\IfDarkModeT,\IfDarkModeF,\IfDarkModeTF}
%    \begin{syntax}
%        \tn{IfDarkModeT} \Arg{true code}
%        \tn{IfDarkModeF} \Arg{false code}
%        \tn{IfDarkModeTF} \Arg{true code} \Arg{false code}
%    \end{syntax}
%    Checks whether the dark mode is enabled or not. If so, \meta{true code} is executed, otherwise \meta{false code} is executed. These macros are fully expandable.
% \end{function}
% \subsection{Commands for developers}
% This functionality is mostly aimed at package developers and template creators, for the end user it should not be necessary to manually fixup the Switching with compatible Packages and Classes.
% \begin{function}[added = 2022-08-30]{\@darkmode@set@enabled,\@darkmode@set@disabled}
%      Functions for template developers that want to implement dark mode toggling with \LaTeXe{}.
% \end{function}
% \end{documentation}
% \clearpage
% \begin{implementation}
%     \section{implementation}
% \begin{variable}{\g_@@_dark_mode_bool}
%    First we define the package Options. The usage of initial ensures that the boolean \cs{g_@@_dark_mode_bool} always exists.
%    \begin{macrocode}
\keys_define:nn {darkmode} {
    enable .bool_gset:N = \g_@@_dark_mode_bool,
    enable .initial:n = false,
    enable .default:n = true,
    defaulthook .bool_gset:N = \g_@@_dark_mode_hook_bool,
    defaulthook .initial:n = true
    defaulthook .default:n = true
    nodefaulthook .meta:n = {defaulthook=false},
}
%    \end{macrocode}
%   Now we process the package options.
%    \begin{macrocode}
\ProcessKeysOptions{darkmode}
%    \end{macrocode}
% \end{variable}

% \begin{macro}[pTF]{\__darkmode_if_dark_mode:}
%    We need a global Conditional to decide whether the dark mode is currently active or not   
%    \begin{macrocode}
\prg_new_conditional:Nnn \__darkmode_if_dark_mode: {T,F,TF} {
    \bool_if:NTF \g_@@_dark_mode_bool
    {\prg_return_true:}
    {\prg_return_false:}
}
%    \end{macrocode}
% \end{macro}
% \begin{macro}{\IfDarkModeT,\IfDarkModeF,\IfDarkModeTF}
%    \begin{macrocode}
\cs_set_eq:NN\IfDarkModeT \__darkmode_if_dark_mode:T
\cs_set_eq:NN\IfDarkModeF \__darkmode_if_dark_mode:F
\cs_set_eq:NN\IfDarkModeTF \__darkmode_if_dark_mode:TF
%    \end{macrocode}
% \end{macro}
% \begin{macro}{\enabledarkmode,\@darkmode@set@enabled,\@darkmode@on@enable}
%    \begin{macrocode}
\definecolor{darkmode@anthrazitgrau}{HTML}{293133}
\colorlet{fgcolor}{.}
\DeclareDocumentCommand{\@darkmode@set@enabled}{}{
    \bool_set_true:c {g_@@_dark_mode_bool} 
}
\DeclareDocumentCommand{\@darkmode@on@enable}{}{
    \pagecolor{darkmode@anthrazitgrau}
    \color{white}
    \selectcolormodel{RGB}
    \colorlet{fgcolor}{.}
}
\DeclareDocumentCommand{\enabledarkmode}{}{
    \@darkmode@set@enabled
    \bool_if:cT {g_@@_dark_mode_hook_bool} {\@darkmode@on@enable}
}
%    \end{macrocode}
% \end{macro}
% \begin{macro}{\disabledarkmode, \@darkmode@set@disabled, \@darkmode@on@disabled}
%    \begin{macrocode}
\DeclareDocumentCommand{\@darkmode@set@disabled}{}{
    \bool_set_false:c {g_@@_dark_mode_bool}
}
\DeclareDocumentCommand{\@darkmode@on@disable}{}{
    \pagecolor{white}
    \color{black}
    \colorlet{fgcolor}{.}
}
\DeclareDocumentCommand{\disabledarkmode}{}{
    \@darkmode@set@disabled
    \bool_if:cT {g_@@_dark_mode_hook_bool} {\@darkmode@on@disable}
}
%    \end{macrocode}
% \end{macro}
% \begin{macro}{\__darkmode_update_from_bool}
%    \begin{macrocode}
\DeclareDocumentCommand{\__darkmode_update_from_bool}{}{
    \IfDarkModeTF{\enabledarkmode}{\disabledarkmode}
}
\__darkmode_update_from_bool
%    \end{macrocode}
% \end{macro}
% \end{implementation}
% \clearpage
% \PrintIndex
\endinput